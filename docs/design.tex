\documentclass{article}
\usepackage[utf8]{inputenx}
\newcommand{\equ}{EQU}
\title{\equ\\
  Diseño y especificación funcional general}
\begin{document}
\maketitle

\section{Introducción}
\label{sec:intro}
\equ\ permite realizar demostraciones, automáticamente verificadas, de
fórmulas ecuacionales.

\section{Caso de uso}
\label{sec:usecase}
\begin{enumerate}
\item El usuario ingresa un \textbf{string} que se intenta \emph{parsear}.
\item Si el parseo fue exitoso entonces producimos un
  \textbf{pre-término} del que se intenta \emph{chequear tipos}. El
  parseo implica reconocer \textbf{operadores}; más allá de la
  representación siempre se debe poder recuperar el parentizado dado.
  Si el parseo falla, entonces se da un mensaje de error informativo.
\item Si el chequeo fue exitoso entonces construimos una
  \textbf{expresión}. El chequeo de tipos consiste básicamente en
  inferir tipos para variables y constantes; se trata de armar el
  árbol de tipado y si se falla entonces se permite que usuario
  encuentre el error guiándolo a que explicité los tipos inconsistentes
  para variables o constantes no-unificables. En este punto también
  se debería permitir que el usuario complete en el árbol de tipado
  todos los tipos. Se podría permitir que el usuario redefina el tipo
  de un operador ``por el momento'' y que se use ese tipo para chequear
  el pre-término; luego se le muestra el tipo real del operador así
  se entiende por qué no se puede tipar la expresión.
\item Dada una expresión se elige una \emph{estrategia} (notar que 
  también es una \textbf{estrategia}) y generamos una \textbf{prueba}.
  Una prueba es una secuencia de pasos de re-escritura. La expresión
  define qué estrategias se pueden aplicar: por ejemplo, si se trata
  de una expresión aritmética sólo podemos aplicar las estrategias de 
  \texttt{evaluación}; si en cambio es una fórmula podemos usar cualquier
  regla (que viene de un axioma o un teorema) para reescribirla.
\item Si todos los pasos son correctos entonces se genera un \textbf{teorema}
  o una nueva regla.
\end{enumerate}

\end{document}
